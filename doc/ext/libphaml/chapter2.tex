%========================================================================
%Beginning of Chapter 2
%========================================================================

\section{THE SOFTWARE COMPONENTS}\label{ch:overview}

\subsection{Introduction}\label{sec:chp2intro}

There are many issues to be addressed when using multiple libraries together.  The scope of the different components presented is quite vast and their dependencies and usage vary extensively.  Understanding the importance and the work involved in tying these together requires a thorough overview of each piece of software.  Some theory will be covered as well to explain what the libraries do or why they're necessary.

%%%%%%%%%%%%%%%%%%%%%%%%%%%%%%%%%%%%%%%%%%%%%%%%%%%%%%%%%%%%%%%%%%%%%%%%%%%%
\subsection{Finite Difference Vs. Finite Element}\label{sec:chp2grid}

Currently many glacial models, including the one used, that run large scale simulations use finite difference methods in order to solve the partial differential equations involved in the numerical approximations.  Finite difference methods are used primarily due to the ease of implementation with fluid dynamics, especially over large areas, and the computations are also less intensive than with finite element methods.

Finite element analysis provides a much more efficient way of calculating the solution with an area that is not of a fixed sized.  The difference can be summed up by a quote from Gershenfeld about finite elements: ``Instead of discretizing the space in which the problem is posed, this (finite element analysis) discretizes the form of the function used to represent the solution." \citep{Gershenfeld:1999}  

As discussed later in chapter \ref{ch:verification} using finite difference methods, both for data and for the calculations, can introduce artifacts and erroneous results if not careful since the domain in ice-sheet modeling can change drastically, but the points at which the approximations are being made do not.


%%%%%%%%%%%%%%%%%%%%%%%%%%%%%%%%%%%%%%%%%%%%%%%%%%%%%%%%%%%%%%%%%%%%%%%%%%%%
\subsection{GLIMMER-CISM}\label{sec:chp2cism}

GLIMMER-CISM is a three-dimensional thermo-mechanical ice-sheet model that is open source and is developed to better understand ice dynamics within glaciers and sea-rise consequences.  GLIMMER (GENIE Land Ice Model with Multiply-Enabled Regions) was originally developed as part of the GENIE (Grid-ENabled Integrated Earth system model) project in order to be the glacial modeling component of the larger model.\citep{glimmerdoc}

Beginning in 2002 GLIMMER was used as the basis for advancing current research of ice-sheet dynamics.  The current model, GLIMMER-CISM (Community Ice Sheet Model) has incorporated many new elements beyond the standard SIA (Shallow Ice Approximation) that the original model used. 
\citep{cismwiki:website}

GLIMMER-CISM now has a higher-order model for mass-balance based on recent ideas by Pattyn \citep{Pattyn:2003}, some verification tools and test suites, more standardization, intermodel comparisons, and a lot of developers building an infrastructure to share data within the community.

GLIMMER-CISM is written primarily in Fortran and uses NetCDF (Network Common Data Form) files as the input and output to the model as well as configuration files for options, model size, simulation lengths, and many other parameters that can be set.  This makes the model very capable and adaptable for both small and large scale simulations. 


%GLIMMER-CISM is based on a staggered grid.  blah

%should i talk about the sia or first order model mathematics?????????
%%%%%%%%%%%%%%%%%%%%%%%%%%%%%%%%%%%%%%%%%%%%%%%%%%%%%%%%%%%%%%%%%%%%%%%%%%%%
\subsection{PHAML}\label{sec:ch2phaml}

PHAML stands for ``Parallel Hierarchical Adaptive MultiLevel method", and it uses several hp-adaptive methods using triangle meshes to solve elliptical partial differential equations.  PHAML incorporates a lot more than just the solving methods though.  The software contains parallel libraries to run parts of the simulation over multiple processors, it has OpenGL bindings allowing the user to receive 3D representations of the solutions, and it is very extendible allowing other solvers to be used.

PHAML is designed to solve linear elliptical partial differential equations of the form:

\begin{equation}
    - \frac{\partial}{\partial x} \left (c_{xx}\frac{\partial u}{\partial x} \right ) - \frac{\partial}{\partial x} \left (c_{xy}\frac{\partial u}{\partial y} \right ) - \frac{\partial}{\partial y} \left (c_{yy}\frac{\partial u}{\partial y} \right ) + c_x \frac{\partial u}{\partial x} + c_y \frac{\partial u}{\partial y} + c_{u}u = f \; in \; \Omega
\label{eq:phamlmain}
\end{equation}

where $c_{xx},c_{yy},c_x,c_y$ and $f$ are functions of $x$ and $y$, and the domain $\Omega$ is a bounded, connected, region in $R^2$.\citep{phamldoc}  The boundary conditions can be Dirchlet, natural, or mixed.  Dirichlet boundary conditions are of the form:

\begin{equation}
u=g \; on \; \partial \Omega_D
\label{eq:dirbound}    
\end{equation}

otherwise the boundary is defined by:
\begin{equation}
    \left ( c_{xx} \frac{\partial u}{\partial x} + c_{xy}\frac{\partial u}{\partial y} \right ) \frac{\partial y}{\partial s} - c_{yy} \frac{\partial u}{\partial y} \frac{\partial x}{\partial s} + c_{bc}u = g \; on \; \partial \Omega_N
\label{eq:phamlbound}
\end{equation}

where $g$ and $c_{bc}$ are also functions of $x$ and $y$ and everything else is as defined for the equation \eqref{eq:phamlmain}.  Here $s$ is some parameter going through some finite range to define the boundary as a function $(x(s),y(s))$.  These equations are explained thoroughly in the PHAML documentation as well as all of the different consequences of a chosen boundary condition. \citep{phamldoc}


%I should talk about lin. elliptical pdes in general.

PHAML works by allowing a custom PSLG (Planar Straight Line Graph) or mesh to be passed to the main program, and then it retrieves any information it needs about the solution (or the problem) by means of subroutine callbacks. These are listed in the appendix \ref{sec:libphamlcall}.  These callbacks allow you to specify any of the coefficients as well as functions of the elliptical PDE.  This makes the solver very flexible since different types of solutions can be found merely be changing one of the callback.  For instance, solving Poisson would have the source subroutine returning the functional value at a given point, but by changing the callback to return zero for all points changes the solution to the Laplacian.

The callbacks also allow for several interesting PDE characteristics to be achieved if desired such as non-homogeneous coefficients, using true solutions, derivative and second derivative based functions, etc.

The mesh geometry can also be generated once the program starts if a functional boundary exists.  There are callbacks that allow the program to define what the boundary is while PHAML is executing.  This is very beneficial when solving for solutions based on a dataset or a function.

One of the most beneficial aspects of PHAML are that it uses hp-adaptive finite element analysis methods.  This allows it to solve PDEs by using piece-wise polynomial approximations that employ elements of variable size (h) and polynomial degree (p). As demonstrated by W. Mitchell this allows hp-FEM methods to not only be more accurate than standard FEM, but that they can also achieve ``a convergence rate that is exponential in the number of degrees of freedom." \citep{mitchell:hp}

Having this amount of flexibility also means that PHAML has a lot of functionality that can be tweaked.  There are many options for nearly every facet of the solving process.  This can be very useful for solutions where a particular type of result is desired.  The options themselves will not be covered since they are listed and described in detail within the PHAML documentation.\citep{phamldoc}



%%%%%%%%%%%%%%%%%%%%%%%%%%%%%%%%%%%%%%%%%%%%%%%%%%%%%%%%%%%%%%%%%%%%%%%%%%%%
\subsection{Triangle}\label{sec:chp2triangle}

Triangle is summed up as ``a two-dimensional quality mesh generator and Delaunay triangulator".\citep{shewchuk96b}  The software is able to read in files describing a two-dimensional graph and build the data structure while adding useful functionality.  The software is also capable of dynamically refining the mesh, adding to the mesh, and providing instant feedback.  Triangle can generate ``exact Delaunay triangulations, constrained Delaunay triangulations, conforming Delaunay triangulations, Voronoi diagrams, and high-quality triangular meshes". \citep{Triangle:website} Most of the algorithms that Triangle depends upon are recent advanced triangulation methods as well as sorting and location algorithms.  Some algorithms stemming from this research, such as the robust geometric predicates, are also now used in many other areas of study. \citep{shewchuk97a}

Triangle is not used directly by the interface, but is used as the mesh component of PHAML.  The GLIMMER-CISM code uses the Triangle program simply to generate the mesh files needed by PHAML.  Since it is needed but not directly used, Triangle is not discussed as thoroughly.

\subsubsection{Showme}

Triangle also comes with a small mesh viewer called `showme' which is very useful for looking at the mesh files directly.  This program opens the .poly files and allows easy access to the properties of the mesh.  This is useful in testing the different generation algorithms by being able to view the output file.  The only drawback of the application is that it doesn't show the bmarks associated with each node. 


%========================================================================
%End of Chapter 2
%========================================================================
