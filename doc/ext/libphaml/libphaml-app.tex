

%code framing environment-------------------------------
%parameter 1 is code to be framed
%displays code framed in a box, single spaced, normal font
\newenvironment{framecode}[1]%
{
 \vspace{.5cm} 
 \begin{center}
 \begin{Sbox}
 \begin{minipage}{#1}

}
{
 \end{minipage}
 \end{Sbox}
 \fbox{\TheSbox}
 \end{center}
 }


%------------------------------------------------------

\newcommand{\appdir}{}

\newenvironment{dispcode}[3]%
{
\section{#1}\label{#3}
\begin{footnotesize}
\verbatiminput{\appdir/#2}
\end{footnotesize}
}
{

}



\chapter{Libphaml}

\section{SETUP}\label{ch:setup}
%%%%%%%%%%%%%%%%%%%%%%%%%%%%%%%%%%%%%%%%%%%%%%%%%%%%%%%%%%%%%%%%%%%%%%%%%%%%%%%%
\section{Introduction}
This appendix presents some compilation notes for the Fortran source files, these can be found in the \href{http://developer.berlios.de/projects/glimmer-cism/}{Glimmer-CISM2} project on the Berlios SVN server.  These include the following files:
\href{http://svn.berlios.de/svnroot/repos/glimmer-cism/glimmer-cism2/libphaml/phaml\_user\_mod.F90}{phaml\_user\_mod.F90},
\href{http://svn.berlios.de/svnroot/repos/glimmer-cism/glimmer-cism2/libphaml/phaml\_support.F90}{phaml\_support.F90},
\href{http://svn.berlios.de/svnroot/repos/glimmer-cism/glimmer-cism2/libphaml/phaml\_pde.F90}{phaml\_pde.F90},
\href{http://svn.berlios.de/svnroot/repos/glimmer-cism/glimmer-cism2/libphaml/phaml\_example.F90}{phaml\_example.F90},
\href{http://svn.berlios.de/svnroot/repos/glimmer-cism/glimmer-cism2/libphaml/phaml\_example\_pde.F90}{phaml\_example\_pde.F90}, and 
\href{http://svn.berlios.de/svnroot/repos/glimmer-cism/glimmer-cism2/libphaml/simple\_phaml.F90}{simple\_phaml.F90}.

These instructions all assume a POSIX-compatible system.  Although all of the software can be compiled on other operating systems, this has not been attempted on any other OS and therefore is absent.  PHAML must be compiled after the graphics libraries if the OpenGL graphics are desired.  The graphic libraries are optional though and unnecessary for running the ice sheet model.  They are merely present for extra visual output if desired.


All projects require make to build.  For the following instructions, the assumption will be made that a global programs directory exists at /usr/bin.  If the target system does not have this, then it is necessary to register the installation directory with the system globally so that other programs can find it.
%%%%%%%%%%%%%%%%%%%%%%%%%%%%%%%%%%%%%%%%%%%%%%%%%%%%%%%%%%%%%%%%%%%%%%%%%%%%%%%%
\section{Triangle}
\subsection{Compiling/Installing}
Triangle must be compiled separate from PHAML and installed.  This process is straightforward.  A program called showme is also compiled and can be installed.  It allows you to view the mesh files that Triangle generates which is useful for testing purposes.

\begin{framecode}{6in}
\begin{verbatim}

make
cp triangle /usr/bin
cp showme /usr/bin

\end{verbatim}
\end{framecode}

%%%%%%%%%%%%%%%%%%%%%%%%%%%%%%%%%%%%%%%%%%%%%%%%%%%%%%%%%%%%%%%%%%%%%%%%%%%%%%%%
\section{PHAML}

\subsection{Getting PHAML}
PHAML can be downloaded as an archive from the website \cite{PHAML:website}, and then unarchived into a directory anywhere on the system.

\subsection{Compiling PHAML}

PHAML is relatively easy to compile if all the library dependencies are satisfied.  The list of dependencies as well as instructions for additional libraries are at the end of this section.  The PHAML user guide \cite{phamldoc} has an excellent section covering setting up some of these libraries and dependencies as well.  The Quickstart guide is a must read before attempting any serious PHAML work.  A lot more detail is also provided on individual software packages that can be used and the benefits they can provide.  These instructions will now assume that the minimum requirements are met.

First the document `mkmkfile.sh' needs to be edited in the root directory of the PHAML source folder.  The following items must be set correctly: DEFAULT\_PHAML\_ARCH, \\ DEFAULT\_PHAML\_OS, DEFAULT\_PHAML\_F90, DEFAULT\_PHAML\_C, \\ DEFAULT\_PHAML\_PARLIB, DEFAULT\_PHAML\_BLAS, and DEFAULT\_PHAML\_LAPACK.  
All other variables can be left to their default value.  The file `mkmkfile.sh' lists the available options for each of these as well.  A brief overview of these are in the dependencies section below.  Once these variables have all been correctly set the file can be closed and run as a script.  As the name suggests, this scripts generates the makefile needed to compile the project with make.

Now just invoke make and it should compile provided things were set correctly.

\begin{framecode}{6in}
\begin{verbatim}

./mkmkfile.sh
make

\end{verbatim}
\end{framecode}

If the code does not compile then either the configuration file is incorrect or a dependency is unsatisfied.


\subsubsection{PHAML Dependencies}
These are required dependencies in order to compile and use PHAML.  
\begin{itemize}
    \item \textbf{POSIX compliance} - An operating system must be Unix compatible in order to properly work with PHAML.  Ubuntu was used for all test work.
    \item \textbf{Make} - The compile system. 
    \item \textbf{A Fortran Compiler} - Most Fortran compilers will work.  GFortran was the compiler it was tested against.
    \item \textbf{A C Compiler} - CC or GCC.  GCC was used.
    \item \textbf{MPI Library} - An MPI server must be running in order for PHAML to communicate with its subprocesses.  Openmpi was chosen for all tests.
    \item \textbf{BLAS} - Compiled from source.
    \item \textbf{LAPACK} - Compiled from source.
    \item \textbf{TRIANGLE} - Compiled from source.  Instructions are above.
\end{itemize}
 

\subsubsection{PHAML Additional Libraries}
PHAML covers all additional libraries in the user guide.  The only extra libraries used in this project were the graphic libraries which have a separate section below with instructions.
\begin{itemize}
    \item \textbf{OpenGL} - The main graphics library.
    \item \textbf{GLUT} - The OpenGL Utility kit.
    \item \textbf{F90GL} - This is a custom interface library so that OpenGL can be called from Fortran.
\end{itemize}

\subsection{Installing PHAML}
PHAML can be installed anywhere on the system.  For a system wide install, using the `opt' folder is convenient, and will be assumed for the instructions.  The PHAML home directory is the top level directory that should have a lib and a modules directory inside of it.  Then the global system variables need to be set.

\begin{framecode}{6in}
\begin{verbatim}

export PHAML_OS=linux
export PHAML_HOME=/opt/phaml

\end{verbatim}
\end{framecode}

\subsection{Including Graphics with PHAML}
\subsubsection{GLUT}

GLUT can be downloaded from the official site or a version can be downloaded from PHAML's website \cite{PHAML:website} that is guaranteed to work with PHAML.  For building this system the one from PHAML was used and it is suggested that this is followed.

Compiling Glut can be somewhat problematic since it depends on OpenGL to compile.  Specifically, it was unable to compile on a Ubuntu system with an ATI graphics card because the open source ATI drivers did not include the older SGIX functions which were needed.  This is largely a vendor issue since the graphics card manufacturer usually supplies the drivers for the card.  When an Nvidia card was used GLUT compiled quickly with no issue since their proprietary drivers included all necessary functions.

The first thing to do is to open Imakefile and remove "test" and "prog" from the SUBDIRS variable.  It should only say ``SUBDIRS = lib".  Then you can run mkmkfiles.imake and compile the project.
\begin{framecode}{6in}
\begin{verbatim}

./mkmkfiles.imake
make

\end{verbatim}
\end{framecode}

Installing GLUT is essential in order to compile F90GL and PHAML with graphics.  This includes installing the libraries and the source header files.  Where OpenGL is installed on a system varies for each operating system.  If the development files for OpenGL are installed on the system, then all the GLUT header files in include/GL should be copied to that OpenGL source directory.  On the test system this was /usr/include/GL.  The all the compiled GLUT libraries need to be copied into the global dynamic library directory.  On most UNIX based systems this is /usr/lib.  Depending on how OpenGL libs are detected the libs might need to also be copied to an X11 directory.


\subsubsection{F90GL}
F90GL can be downloaded from the official site which is the same as PHAML's website \cite{PHAML:website}.  The package can then be unarchived and compiled anywhere on the system.

Compiling F90GL is not that complicated, but it takes more time to set up than GLUT does.  The package includes custom compile scripts based on architecture, operating system, Fortran compiler, and OpenGL version.  The file that lists what the abbreviations are for is `mf\_key'.  Simply use the file that relates to the intended system and then run make while specifying the system.  The other important note is that the script may need to be modified to point to the correct GLUT libraries that were just compiled.  

\begin{framecode}{6in}
\begin{verbatim}

make -f mflum2

\end{verbatim}
\end{framecode}

%%%%%%%%%%%%%%%%%%%%%%%%%%%%%%%%%%%%%%%%%%%%%%%%%%%%%%%%%%%%%%%%%%%%%%%%%%%%%%%%
\section{GLIMMER-CISM}

There is fairly good documentation for working with GLIMMER on different platforms and therefore the documentation presented here will be focused on compiling GLIMMER-CISM from the repositories as it was done on the test build.

\subsection{Compiling GLIMMER-CISM}
Compiling GLIMMER is fairly straightforward and relies on the MAKE build system like the other projects have.  The dependencies are listed below and should be satisfied before attempting to compile GLIMMER-CISM itself.  Once these are installed GLIMMER-CISM2 can be checked out from the repositories. \citeauthor{cism:website}

\subsubsection{GLIMMER-CISM Dependencies}
    \begin{itemize}
        \item \textbf{Autoconf} - Tool used in the build system.
        \item \textbf{Make} - The compile system. 
        \item \textbf{A Fortran Compiler} - Most Fortran compilers should work.  GFortran was the compiler it was tested against.
        \item \textbf{A C Compiler} - CC or GCC.  GCC was used.
        \item \textbf{NetCDF} - The libraries for reading a NetCDF file.
        \item \textbf{Python} - Many build scripts and drivers use python scripting.
    \end{itemize}

\subsubsection{GLIMMER-CISM Additional Libraries}
GLIMMER-CISM is a very robust system with the ability to add several different solvers, libraries, and components.  They will not be listed here, but more information can be found in the user guide. \cite{glimmerdoc}

In general compiling GLIMMER-CISM proceeds as such:
\begin{itemize}
    \item Check out the project \href{http://svn.berlios.de/svnroot/repos/glimmer-cism/glimmer-cism2/}{GLIMMER-CISM2} using svn or get an archived file of the project for download.
    \item Open a terminal in the directory of the project.
    \item Run the following commands:

\begin{framecode}{6in}
\begin{verbatim}

./bootstrap 
./configure --with-netcdf=/usr FCFLAGS="-DNO_RESCALE 
            -O3 -pedantic-errors -fbounds-check" 
make  

\end{verbatim}
\end{framecode}

\end{itemize}
The ``--with-netcdf=/usr" can be omitted if the build system can find NetCDF in it's default location on the system.
\citep{cism:website}

\subsection{Installing GLIMMER-CISM}
Provided everything compiled correctly the build system has a built-in method for installing the binaries on the system.  Root privileges may be required. 
\begin{framecode}{6in}
\begin{verbatim}

make install

\end{verbatim}
\end{framecode}
%%%%%%%%%%%%%%%%%%%%%%%%%%%%%%%%%%%%%%%%%%%%%%%%%%%%%%%%%%%%%%%%%%%%%%%%%%%%%%%%
\section{Compiling PHAML With GLIMMER-CISM}
There are a few changes to the build system in order to provide the ability for PHAML to compile with GLIMMER-CISM as well as a few additional options that are available to debug the system.

%--with-phaml=/phaml
%--with-phaml-graphics=/f90gl do I need glut specified
After bootstrapping GLIMMER-CISM the same commands need to be run but now specifying to use phaml, and if debugging the OpenGL graphics.  Note that the graphics are not needed and are merely for extra visualization during debugging, but there is a lot of overhead required in getting libraries working. 

In order to compile PHAML by itself with GLIMMER-CISM you want:
\begin{framecode}{6in}
\begin{verbatim}

./bootstrap 
./configure --with-netcdf=/usr --with-phaml=/opt/phaml 
            FCFLAGS="-DNO_RESCALE -O3 -pedantic-errors -fbounds-check"
make  
make install

\end{verbatim}
\end{framecode}

%"LDFLAGS= -lphaml"
In order to include the graphics as well you'll need to add the optional configure tags as well as the libraries to include.  The graphics tag needs the location of the F90GL files.  The OpenGL libraries should automatically be linked by a system-wide install.

\begin{framecode}{6in}
\begin{verbatim}

./bootstrap 
./configure --with-netcdf=/usr --with-phaml=/opt/phaml 
            --with-phaml-graphics=/opt/f90gl FCFLAGS="-DNO_RESCALE 
            -O3 -pedantic-errors -fbounds-check" 
make 
make install

\end{verbatim}
\end{framecode}

%"LDFLAGS=-lf90glut -lf90GLU -lf90GL -lglut -lGLU -lGL -lphaml" 


\section{GLIMMER-CISM/PHAML USAGE}\label{ch:usage}
%%%%%%%%%%%%%%%%%%%%%%%%%%%%%%%%%%%%%%%%%%%%%%%%%%%%%%%%%%%%%%%%%%%%%%%%%%%%%%%%
\subsection{Example Code}\label{sec:examplecode}

The example code for PHAML demonstrates calling the phaml\_xxxx modules from within the ice-sheet model and how to start new solutions.  This is the standard way most of the libraries for GLIMMER-CISM are used.  This requires loading everything the model requires and running through the full set of calculations on the ice-sheet.  The calls to PHAML will be one small part of the overall simulation process.  Therefore, the example code is demonstrative of the smaller piece of code that would be in a larger module.


\begin{framecode}{6in}
\begin{verbatim}

use phaml
use phaml_example
use glide_types
type(phaml_solution_type) :: phaml_solution
type(glide_global_type) :: cism_model

!initialize all variables needed
call phaml_init(cism_model,phaml_solution)

!does the evaluation and places the 
!solution in cism_model%phaml%uphaml
call phaml_evolve(cism_model,phaml_solution)

!close and free variables
call phaml_close(phaml_solution)

\end{verbatim}
\end{framecode}

This is an example of when only the solution is desired and no intermediate steps are needed.  If a nonlinear or relaxation type of simulation is required then it is possible to walk through the solution one step at a time like the example below.  All of the options in PHAML can be tweaked in the evolve procedures of each module if needed.

\begin{framecode}{6in}
\begin{verbatim}

use phaml
use phaml_example
use glide_types
type(phaml_solution_type) :: phaml_solution
type(glide_global_type) :: cism_model

!initialize all variables needed
call phaml_init(cism_model,phaml_solution)

!creates the mesh and sets initial conditions 
call phaml_setup(cism_model,phaml_solution)

!looping through timesteps
do while(time .le. model%numerics%tend)

    !copy old solution and do one iteration
    call phaml_nonlin_evolve(cism_model,phaml_solution)
    
    !get the solution and copy to desired variable
    call phaml_getsolution(phaml_solution, cism_model%phaml%uphaml)
end do

!close and free variables
phaml_close(phaml_solution)

\end{verbatim}
\end{framecode}
%%%%%%%%%%%%%%%%%%%%%%%%%%%%%%%%%%%%%%%%%%%%%%%%%%%%%%%%%%%%%%%%%%%%%%%%%%%%%%%%
\subsection{Standalone Code}

GLIMMER-CISM provides an excellent framework for doing small scale simulations by using a basic set of libraries.  This is a good way to work with PHAML as well since you can integrate it with GLIMMER-CISM for simple tests without the overhead of the entire model.  There is an simple example driver like this included in the libphaml source files aptly named \href{http://svn.berlios.de/svnroot/repos/glimmer-cism/glimmer-cism2/libphaml/simple\_phaml.F90}{simple\_phaml.F90}. 

%%%%%%%%%%%%%%%%%%%%%%%%%%%%%%%%%%%%%%%%%%%%%%%%%%%%%%%%%%%%%%%%%%%%%%%%%%%%%%%%
\subsection{Debugging Options}

PHAML provides options to hand running code over to a debugger so that slaves can be monitored separately from the master.  These are very useful when handling usermod variables or when using many slaves on different processors.  

When calling `phaml\_create' the parameter `spawn\_form=DEBUG\_SLAVE' can be passed and then slaves will spawn in a an xterm window with a debugger.  This requires compiling with the `-g' flag though, and will default to GDB for the debugger.  If a different debugger is desired you can set the `debug\_command' parameter to specify which to use.


\section{ADDING MODULES/DRIVERS} \label{ch:addingmods}
%%%%%%%%%%%%%%%%%%%%%%%%%%%%%%%%%%%%%%%%%%%%%%%%%%%%%%%%%%%%%%%%%%%%%%%%%%%%%%%%
\section{Introduction}\label{sec:addintro}

The build system for GLIMMER-CISM is extensive and great care needs to be taken in order to ensure that everything is compiled correctly.  This appendix demonstrates how a new phaml\_module.F90 file and phaml\_module\_pde.F90 file can be added to the build system after they have been created.

%%%%%%%%%%%%%%%%%%%%%%%%%%%%%%%%%%%%%%%%%%%%%%%%%%%%%%%%%%%%%%%%%%%%%%%%%%%%%%%%
\section{Adding Modules}\label{sec:addmod}

When adding a module the file `Makefile.am' in the libphaml directory will need to be edited in order to add the new file to the build process and tell the system the necessary dependencies the module requires.  We'll assume the file to be edited is named `phaml\_module.F90'.  The first things that will have to happen is adding the library to be created to the `lib\_LTLIBRARIES' variable at the top of the `Makefile.am' file.  It will look like:

\begin{framecode}{6in}
\begin{verbatim}

lib_LTLIBRARIES = ......... libphaml_module.la

\end{verbatim}
\end{framecode}

The next thing that must be added is the compile instructions for the new library.  Add a block under the other existing library instructions like this:

\begin{framecode}{6in}
\begin{verbatim}

#new phaml library xxxx to be used by glimmer
libphaml_module_la_SOURCES = phaml_module.F90 
libphaml_module_la_LIBADD = libphaml_user_mod.la libphaml_support.la \
    #add additional libraries needed here

\end{verbatim}
\end{framecode}

Make sure libraries are not being added multiple times.  If another library uses this module not all libraries will need to be added again.  If the new library uses any GLIMMER or GLIDE libraries those don't need to be added here because they are included when the binary is built.


And finally in order for the pde callbacks to be available the file phaml\_module\_pde.F90 will need to be added to the phaml\_slave\_SOURCES variable as well.  The module will also need to be added to the \href{http://svn.berlios.de/svnroot/repos/glimmer-cism/glimmer-cism2/libphaml/phaml\_pde.F90}{phaml\_pde.F90}
 functions for the module to be used.  This requires adding a `use' statement in each subroutine like so: 

\begin{framecode}{6in}
\begin{verbatim}

use phaml_module_pde

\end{verbatim}
\end{framecode}

%%%%%%%%%%%%%%%%%%%%%%%%%%%%%%%%%%%%%%%%%%%%%%%%%%%%%%%%%%%%%%%%%%%%%%%%%%%%%%%%
\section{Adding Drivers}\label{sec:adddrive}

Sometimes it might be desired to add a driver as a very simplistic version of the GLIMMER-CISM model where only certain portions of the model are used in order to test a new PHAML module.  This process is outlined by the `simple\_phaml' binary that is built.  Like the libraries, the binary must first be added to the list of binaries to create.  The assumption is made that there exists a file phaml\_driver.F90 being used for this program.

\begin{framecode}{6in}
\begin{verbatim}

bin_PROGRAMS = ........ phaml_driver

\end{verbatim}
\end{framecode}

Now the binary can be defined by what source files as well as what libraries it needs in order to compile.  The order in which the libraries are listed is important.  If library A is needed by library B, then the order must be A,B in the library listing.

\begin{framecode}{6in}
\begin{verbatim}

phaml_driver_SOURCES = phaml_driver.F90 
phaml_driver_LDADD = $(ac_cv_phaml_prefix)/lib/libphaml.a \
    $(top_builddir)/libglide/libglide.la \
    $(top_builddir)/libglimmer-solve/libglimmer-solve.la \
    $(top_builddir)/libglimmer/libglimmer-IO.la \
    $(top_builddir)/libglimmer/libglimmer.la \
    $(NETCDF_LDFLAGS) $(NETCDF_LIBS) $(MPILIBS) \
    libphaml_user_mod.la libphaml_example.la libphaml_pde.la \
    libphaml_module.la
    #add additional libraries needed here

\end{verbatim}
\end{framecode}

That is everything needed in the Makefile.am file.  Now CISM must be rebuilt starting with the bootstrap and configured with the ``--with-phaml" option.



\section{PHAML NETCDF VARIABLES} \label{ch:phamlvars}

These variables are contained within the glide\_phaml custom type that resides within the glimmer model global type.  The load identifier is whether or not GLIMMER-CISM can load this variable from a NetCDF input file.

\begin{table}[ht]
\caption{PHAML Type Variables}
\centering
\begin{center}
    \begin{tabular}{ | c | c | c |}
    \hline
    \multicolumn{3}{|c|}{PHAML PDE Functions}\\
    \hline
    Variable Name & Description & Load?\\ \hline
    uphaml & The true solution if known & Y\\ \hline
    init\_phaml & Initial conditions of the PDE domain & Y\\ \hline
    rs\_phaml & The source values if static & N \\ \hline
\end{tabular}
\end{center}
\end{table}

\section{LIBPHAML FUNCTIONS} \label{ch:libphamlfunc}

This appendix describes the functions that are provided in the phaml\_example module and that must be maintained in any new modules.  The subroutines exist within the example module rather than the support module so that they can be modified or tweaked based on the specific problem.  PHAML provides many options to all function calls and this method allows the options to be different between modules if desired without affecting another module.

These are wrappers to ease the use of PHAML and to setup all initial conditions and make sure everything is properly handled with PHAML and the other modules needed.  Please see the guide for the native PHAML functions. \citep{phamldoc} 

\subsection{Main Module}\label{sec:libfuncmain}

\begin{itemize}
\item \textbf{phaml\_init} - This subroutine simply sets the needed variables for the usermod module to work.  It does not instantiate the phaml\_solution.
\item \textbf{phaml\_setup} - If doing a non-linear PDE problem then this initializes the phaml\_solution, creates the mesh, and sets the initial conditions.  To solve phaml\_nonlin\_evolve must be called.
\item \textbf{phaml\_evolve} - This is a single pass solve where the function assumes the problem is linear.  It creates the mesh, initializes PHAML, solves the problem, retrieves the solution, then closes PHAML.
\item \textbf{phaml\_nonlin\_evolve} - This subroutine assumes phaml\_setup has already been called and that the solution is incremental.  It copies the old solution then does another iteration and returns.
\item \textbf{phaml\_getsolution} - Given the phaml\_solution variable it simply returns the current solution at the node points of the GLIMMER-CISM model grid. 
\item \textbf{phaml\_close} - This destroys the phaml session variable as well as deallocates the variables used in usermod.
\end{itemize}

\subsection{PHAML Callbacks}\label{sec:libphamlcall}

This subsection lists the subroutines and functions that PHAML relies on and must be present in order for it to define the PDE and to find a solution.  Given the type of solution desired, many of them don't need to be used, but they must all still exist even if returning zero.  Please refer to the manual for more detailed descriptions, special circumstances, arguments, and examples.  \citep{phamldoc}

\begin{itemize}
\item \textbf{pdecoefs} - This subroutine returns the coefficient and right hand side of the PDE at the point (x,y). 
\item \textbf{bconds} - This subroutine returns the boundary conditions at the point (x,y).
\item \textbf{iconds} - This routine returns the initial condition for a time dependent problem at the point (x,y).
\item \textbf{trues} - This is the true solution of the differential equation at point (x,y), if known.
\item \textbf{truexs} - This is the x derivative of the true solution of the differential equation at point (x,y), if known.
\item \textbf{trueys} - This is the y derivative of the true solution of the differential equation at point (x,y), if known. 
\item \textbf{boundary\_point} -  This routine defines the boundary of the domain at the point (x,y) if no mesh was provided. 
\item \textbf{boundary\_npiece} - This routine gives the number of pieces in the boundary definition if no mesh was provided. 
\item \textbf{boundary\_param} - This routine gives the range of parameter values for each piece of the boundary if no mesh was provided.
\item \textbf{phaml\_integral\_kernel} - This is the identity function that PHAML requires and shouldn't need any modification.
\item \textbf{regularity} - Provides the \emph{a priori} knowledge about the singular nature of the solution if applicable.
\item \textbf{update\_usermod} - This routine updates the module variables by sending them from the master to the slave processes.  This function is very important for the simulation to work correctly, and the data formatting is addressed in more detail in section \ref{sec:libfuncuser}.

\end{itemize}
%\begin{table}[ht]
%\caption{PHAML PDE Callback Functions \cite{phamldoc}}
%\centering
%\begin{center}
%    \begin{tabular}{ | c | c |}
%    \hline
%    \multicolumn{2}{|c|}{PHAML PDE Functions}\\
%    \hline
%    Function Name & Description \\ \hline
%    pdecoefs & This subroutine returns the coefficient and right \\
%    &   hand side of the PDE at the point (x,y). \\ \hline
%    bconds & This subroutine returns the boundary conditions at the\\
%    &   point (x,y). \\ \hline
%    iconds & This routine returns the initial condition for a time\\
%    &   dependent problem at the point (x,y). \\ \hline
%    trues & This is the true solution of the differential equation\\
%    &   at point (x,y), if known.\\ \hline
%    truexs & This is the x derivative of the true solution of the \\
%    &   differential equation at point (x,y), if known. \\ \hline
%    trueys & This is the y derivative of the true solution of the \\
%    &   differential equation at point (x,y), if known. \\ \hline
%    boundary\_point &  This routine defines the boundary of the domain \\
%    &   at the point (x,y) if no mesh was provided. \\ \hline
%    boundary\_npiece & This routine gives the number of pieces in the boundary\\
%    &   definition if no mesh was provided. \\ \hline
%    boundary\_param & This routine gives the range of parameter values for\\
%    &   each piece of the boundary if no mesh was provided. \\ \hline
%    \end{tabular}
%\end{center}
%\end{table}

\subsection{Support Module}\label{sec:libfuncsupp}

These subroutines are independent of PHAML and are simply support functions needed by the various PHAML modules that can be created.  There is a possibility that some of them may need to be modified depending on a particular simulation need, but in general should be applicable to most situations.

\begin{itemize}
\item \textbf{is\_ice\_edge} - This function uses the mask to determine if a node is the last node on the glacier to have ice.  It does this by checking all four surrounding nodes to make sure at least one of them doesn't have ice. 
\item \textbf{get\_bmark} - This function returns the boundary marker required in the .poly file for edges.  Currently it uses the `mask' value, but can be changed depending on other needs.
\item \textbf{make\_ice\_poly\_file} - This subroutine generates the mesh file that PHAML loads by only using nodes that have ice as decided by the mask in CISM.  It uses the get\_bmark and is\_ice\_edge subroutines.  Once it writes out the .poly file it calls Triangle in order to process it for use by PHAML.  
\item \textbf{make\_full\_poly\_file} - This subroutine generates the mesh file that PHAML loads by using the full domain space.  A rectangular grid will always be output.  The function writes out the .poly file and then calls Triangle in order to process it for use by PHAML.  
\end{itemize}

\subsection{Usermod Module}\label{sec:libfuncuser}

The usermod module is a set of variables and routines that are used or might be used from within the PHAML callbacks.  Any data coming from GLIMMER-CISM would need to be set in one of these variables and then could be used in a callback.  All the data must be passed on to PHAML's slaves though the function `update\_usermod' that is in the set of PHAML callbacks.  These callbacks can be tricky and are explained further in section \ref{sec:libphamlcall}.  The usermod module is addressed in chapter \ref{ch:softintegration} section \ref{sec:ch4usermod}.

\subsubsection{The functions}

\begin{itemize}
\item \textbf{user\_init} - This function sets up the initial data for the usermod module.
\item \textbf{user\_close} - This function serves to call any closing subroutines or deallocate any user data that was initialized.
\item \textbf{array\_init} - Once the slaves have the data from user\_init, the array variables can be allocated and sent as well.  This function should be called from within update\_usermod.
\item \textbf{array\_close} - This is to deallocate the arrays used in the usermod function which had to be dynamically allocated.
\item \textbf{concat\_arrays} - Since the usermod requires all data be passed in one array it might be necessary to pass more than one, and it would be easy to lose track of them.  This function concatenates them together in a consistent fashion.
\item \textbf{split\_arrays} - This is the inverse function to concat\_arrays.  It will split an array back into the original two based on the length specified in the usermod data. 
\item \textbf{reshape\_array\_to\_one} - In order to pass data to the slaves in PHAML, all data must be passed in a single dimension array.  This function takes a two dimensional array and converts it to one dimension.
\item \textbf{reshape\_array\_to\_two} - This is the inverse function to reshape\_array\_to\_one.  It takes in the single dimension array and splits it back into two dimensions based on the model `nsn' and `ewn'.
\item \textbf{get\_xyarrays} - PHAML returns the solution in one long array, so this function returns two arrays with the corresponding node locations in absolute dimensions to pass to the phaml\_evaluate function.
\item \textbf{getew} - Given an x in absolute coordinates, this divides it by the `dew' and truncates it into an integer in order to return the nearest ew coordinate for the grid.
\item \textbf{getns} - Given a y in absolute coordinates, this divides it by the `dns' and truncates it into an integer in order to return the nearest ns coordinate for the grid.
\end{itemize}

\subsubsection{The variables}

\begin{itemize}
\item \textbf{gnsn} - The number of nodes in the north/south direction of the grid.
\item \textbf{gewn} - The number of nodes in the east/west direction of the grid.
\item \textbf{gdns} - The representative distance in meters between each node in the north / south direction. 
\item \textbf{gdew} - The representative distance in meters between each node in the east / west direction.
\item \textbf{num\_arrays} - The number of arrays needed to be passed via update\_usermod
\item \textbf{modnum} - The unique identifier (integer) for this module so that the correct callback functions will be used.
\end{itemize}

